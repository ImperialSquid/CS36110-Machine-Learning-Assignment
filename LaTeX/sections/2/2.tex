\section{2}

The dataset given with this assignment has several missing values in it, namely 46 (7\%) in Reason\_for\_absence and 3 (less than 0.5\%) in Month\_\\of\_absence. There are a number of ways to classify missing values, namely, MCAR or Missing Completely At Random (there is no relationship between the fact a value is missing and the rest of the instance), MAR or Missing at Random (there is a relationship between which values are missing and the rest of their instances but not what the value is) and MNAR or Missing Not At Random (the values which are missing and their actual value are linked to their instances). 

I imported the dataset into R Studio so I could analyse the data (using the package \cite{packfarff}). If the Reason\_of\_absence was MCAR the distribution of each feature for both missing and non-missing reasons would be the same, however we see this is not the case (for conciseness only a few examples are given here, however all the plots made can be found in the appendices, plots were made with \cite{packggplot2}). As can be observed there is a vast difference between the distributions of each feature depending on if there's an associated reason for the absence. 

The fact that the missing data is not MCAR, it represents a significant proportion of our dataset and removing it would change the distributions of our features, we must therefore impute the data \citep{cheema2014}. Given the data was already R, I used a package called mice (Multivariate Imputation by Chained Equations) \cite{packmice}. mice works by replacing each variable with a starting replacement (usually the mean or mode) it then 